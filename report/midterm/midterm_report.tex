\documentclass[12pt,a4paper]{ctexart}
\usepackage{geometry}
\usepackage{graphicx}
\usepackage{xcolor}
\usepackage{titlesec}
\usepackage{hyperref}
\usepackage{booktabs}
\usepackage{longtable}
\usepackage{float}
\usepackage{caption}
\usepackage{amsmath}
\usepackage{amssymb}
\usepackage{array}
\usepackage{tabularx}
\usepackage{makecell}
\usepackage{fancyhdr}
\usepackage{listings}
\usepackage{enumitem}
\usepackage[most]{tcolorbox}

% 页面设置
\geometry{left=2.5cm,right=2.5cm,top=3cm,bottom=2.5cm,headheight=15pt}

% 页眉页脚设置
\pagestyle{fancy}
\fancyhf{}
\fancyhead[L]{人工智能与社会课程大作业}
\fancyhead[R]{中期报告}
\fancyfoot[C]{\thepage}
\renewcommand{\headrulewidth}{0.4pt}
\renewcommand{\footrulewidth}{0pt}

% 超链接设置
\hypersetup{
    colorlinks=true,
    linkcolor=blue,
    filecolor=magenta,      
    urlcolor=cyan,
}

% 标题格式设置
\titleformat{\section}{\Large\bfseries\color{blue!70!black}}{\thesection}{1em}{}
\titleformat{\subsection}{\large\bfseries}{\thesubsection}{1em}{}

% 代码样式
\lstset{
    basicstyle=\small\ttfamily,
    breaklines=true,
    frame=single,
    backgroundcolor=\color{gray!10},
    numbers=left,
    numberstyle=\tiny\color{gray},
    keywordstyle=\color{blue},
    commentstyle=\color{green!60!black},
    stringstyle=\color{red!70!black},
    showstringspaces=false,
    tabsize=4
}

% Python 代码样式
\lstdefinestyle{python}{
    language=Python,
    morekeywords={self, True, False, None, async, await, with, as},
}

% 文档信息
\title{\textbf{\Huge 人工智能与社会课程\\[0.3cm]大作业中期报告}\\[1cm]
\Large 基于 Casevo 框架的智能体决策能力优化研究}
\author{}
\date{}

\begin{document}

\maketitle
\thispagestyle{empty}

\vspace{0.5cm}

\begin{center}
\renewcommand{\arraystretch}{1.5}
\begin{tabular}{@{}rl@{}}
\textbf{小组成员:} & \large 王宇东(组长)、陈文远 \\
\textbf{提交日期:} & \large \today \\
\end{tabular}
\end{center}

\vspace{1cm}

\begin{abstract}
\noindent
本中期报告总结了"基于 Casevo 框架的智能体决策能力优化研究"项目的当前进展。截至目前,我们已完成四个核心优化模块的开发,包括增强型思维链、高级记忆管理、协同决策和决策评估模块,并成功运行了选举投票、资源分配和信息传播三个实验场景的完整对比实验。

实验结果表明,在选举模拟场景中,gpt-4o 模型以 51.5\% 的 Biden 支持率几乎完美复现了 2020 年美国大选结果(实际为 51.3\%),偏差仅为 0.2\%。在推理机制对比实验中,Tree of Thought(ToT)相比 Chain of Thought(CoT)在中间派选民的决策一致性上提升了 10\%。此外,LLM 驱动的资源分配协商产生了更公平的分配方案,基尼系数相比规则型基线降低了 9.4\%。本报告详细记录了系统实现细节、实验配置和结果分析。

\vspace{0.5em}
\noindent\textbf{关键词:}大语言模型;多智能体系统;社会模拟;思维链;决策优化
\end{abstract}

\newpage
\setcounter{page}{1}
\tableofcontents
\newpage

%========================================
\section{项目概述}
%========================================

\subsection{研究目标}

本项目旨在对 Casevo(Cognitive Agents and Social Evolution Simulator)框架进行系统性优化,提升智能体在复杂社会模拟场景中的决策质量和适应性。研究聚焦于四个核心优化方向:首先,在推理机制层面,将原有的线性思维链(Chain of Thought, CoT)扩展为树状思维(Tree of Thought, ToT),支持多路径探索和分支评估;其次,在记忆管理层面,引入上下文感知检索、时间衰减权重和智能遗忘机制,提升记忆检索的相关性;第三,在反思机制层面,设计基于置信度的自适应反思触发策略,使智能体能够在决策不确定时主动进行深度思考;最后,在协同决策层面,构建支持分布式协商和集中式聚合的多智能体决策框架。

\subsection{总体进度}

\begin{table}[H]
\centering
\caption{项目进度总览}
\renewcommand{\arraystretch}{1.3}
\begin{tabular}{@{}llcc@{}}
\toprule
\textbf{阶段} & \textbf{任务} & \textbf{计划周次} & \textbf{完成度} \\
\midrule
第一阶段 & 文献调研与方案设计 & 第6--7周 & 100\% \\
第二阶段 & 核心模块开发 & 第8--11周 & 100\% \\
第三阶段 & 实验设计与数据采集 & 第12--14周 & 80\% \\
第四阶段 & 对比实验与分析 & 第15--16周 & 进行中 \\
第五阶段 & 报告撰写 & 第17周 & 待开始 \\
\bottomrule
\end{tabular}
\end{table}

%========================================
\section{系统实现}
%========================================

\subsection{系统架构}

图~\ref{fig:architecture} 展示了优化后的 Casevo 框架整体架构,包含四个核心优化模块及其交互关系。

\begin{figure}[H]
\centering
\includegraphics[width=0.85\textwidth]{figure/system.jpg}
\caption{优化后的 Casevo 框架架构图}
\label{fig:architecture}
\end{figure}

\subsection{模块开发完成情况}

根据 Proposal 的技术路线,我们已完成以下核心模块的开发:

\begin{table}[H]
\centering
\caption{核心模块开发状态}
\renewcommand{\arraystretch}{1.3}
\begin{tabular}{@{}lll@{}}
\toprule
\textbf{模块名称} & \textbf{文件路径} & \textbf{状态} \\
\midrule
增强型思维链 (ToT) & \texttt{enhanced\_chain.py} & $\checkmark$ 完成 \\
高级记忆管理 & \texttt{advanced\_memory.py} & $\checkmark$ 完成 \\
协同决策框架 & \texttt{collaborative\_decision.py} & $\checkmark$ 完成 \\
决策质量评估 & \texttt{decision\_evaluator.py} & $\checkmark$ 完成 \\
OpenAI LLM 接口 & \texttt{llm\_openai.py} & $\checkmark$ 完成 \\
\bottomrule
\end{tabular}
\\[0.3em]
\small\textit{注:所有模块位于 \texttt{src/casevo/} 目录下}
\end{table}

\subsection{Tree of Thought 实现}

树状思维模块的设计目标是支持多路径探索和分支评估,其实现包含四个核心类。\texttt{ToTNode} 类作为思维树的基本节点,负责存储当前路径的状态信息和累积评分。\texttt{ToTStep} 类实现分支生成逻辑,在每个决策点调用 LLM 生成多个候选思维路径。\texttt{EvaluatorStep} 类负责评估各分支的质量并执行剪枝操作,保留最具潜力的探索方向。\texttt{TreeOfThought} 类作为顶层控制器,整合上述组件,实现完整的探索、评估和回溯流程。

\subsection{高级记忆系统}

记忆系统的优化涵盖四个关键特性。在时间衰减机制方面,记忆的重要性权重随时间推移呈指数衰减,其数学表达式为 $w(t) = e^{-\lambda(t_{now} - t_{create})}$,其中 $\lambda$ 为衰减系数,$t_{now}$ 和 $t_{create}$ 分别表示当前时间和记忆创建时间。在检索策略方面,系统采用上下文感知检索机制,综合考虑语义相似度、时间相关性和情境匹配度三个维度进行记忆召回。此外,系统实现了智能遗忘功能,当记忆数量超过预设阈值时,自动识别并删除低重要性记忆,以维持记忆库的高效运作。最后,记忆压缩机制能够自动合并语义相似的记忆条目,有效减少存储冗余。

\subsection{Prompt 模板}

为支持各种决策场景,我们设计了以下 Jinja2 模板:

\begin{table}[H]
\centering
\caption{Prompt 模板列表}
\renewcommand{\arraystretch}{1.3}
\begin{tabularx}{\textwidth}{@{}l>{\raggedright\arraybackslash}X@{}}
\toprule
\textbf{模板文件} & \textbf{用途} \\
\midrule
\texttt{tot\_generate.j2} & ToT 分支生成,要求 LLM 输出多个候选方案 \\
\texttt{tot\_evaluate.j2} & ToT 分支评估,对候选方案进行打分 \\
\texttt{reflect\_dynamic.j2} & 动态反思,基于置信度触发深度思考 \\
\texttt{negotiate.j2} & 多智能体协商,生成协商策略和立场调整 \\
\bottomrule
\end{tabularx}
\end{table}

\subsection{核心代码示例}

以下展示 Tree of Thought 模块的核心实现逻辑:

\begin{lstlisting}[style=python, caption={ToT 分支生成与评估核心代码}, label={lst:tot}]
class TreeOfThought:
    def __init__(self, llm, max_branches=3, max_depth=3):
        self.llm = llm
        self.max_branches = max_branches
        self.max_depth = max_depth
    
    def explore(self, question: str, context: dict) -> str:
        """执行树状思维探索"""
        root = ToTNode(state=context, depth=0)
        best_path = self._dfs_explore(root, question)
        return best_path.final_answer
    
    def _dfs_explore(self, node: ToTNode, question: str):
        if node.depth >= self.max_depth:
            return node
        
        # 生成多个候选分支
        branches = self._generate_branches(node, question)
        
        # 评估并剪枝
        scored = [(b, self._evaluate(b, question)) for b in branches]
        scored.sort(key=lambda x: x[1], reverse=True)
        
        # 递归探索最优分支
        best = scored[0][0]
        return self._dfs_explore(best, question)
\end{lstlisting}

%========================================
\section{实验设计与结果}
%========================================

\subsection{实验场景概述}

根据 Proposal 规划,我们设计了三个实验场景:

\begin{table}[H]
\centering
\caption{实验场景配置}
\renewcommand{\arraystretch}{1.3}
\begin{tabularx}{\textwidth}{@{}lcl>{\raggedright\arraybackslash}X@{}}
\toprule
\textbf{场景} & \textbf{智能体数} & \textbf{轮次} & \textbf{评估指标} \\
\midrule
选举投票 & 101 & 6 & 投票分布、与真实选举偏差 \\
资源分配 & 30 & 4 & 基尼系数、满足率、收敛速度 \\
信息传播 & 50 & 6 & 判断准确率、虚假信息识别率 \\
\bottomrule
\end{tabularx}
\end{table}

%----------------------------------------
\subsection{实验一:选举投票模拟}
%----------------------------------------

\subsubsection{实验配置}

本实验构建了包含 101 位选民智能体的选举模拟系统。智能体间的社交网络采用 Watts-Strogatz 小世界网络模型,平均度数设置为 4,重连概率为 0.3。选民的政治画像基于 Pew Research Center 的政治类型学研究,涵盖 9 类不同政治倾向的选民群体。实验共进行 6 轮辩论模拟,议题依次为经济政策、边境安全、疫情应对、法律秩序、宗教自由和对华贸易。为评估不同 LLM 的模拟能力,实验分别使用 gpt-4o-mini 和 gpt-4o 两个模型进行对比测试。

\subsubsection{实验结果}

\begin{figure}[H]
\centering
\includegraphics[width=0.85\textwidth]{figure/fig_model_comparison.png}
\caption{不同模型的选举模拟结果对比}
\label{fig:model-comparison}
\end{figure}

\begin{table}[H]
\centering
\caption{选举模拟最终投票结果(101 位选民)}
\renewcommand{\arraystretch}{1.3}
\begin{tabular}{@{}lccc@{}}
\toprule
\textbf{模型} & \textbf{Biden 票数} & \textbf{Trump 票数} & \textbf{Biden 得票率} \\
\midrule
gpt-4o-mini & 59 & 42 & 58.4\% \\
gpt-4o & 52 & 46 & \textbf{51.5\%} \\
2020 真实结果 & --- & --- & 51.3\% \\
\bottomrule
\end{tabular}
\end{table}

实验结果揭示了若干重要发现。gpt-4o 模型的模拟结果(Biden 得票率 51.5\%)与 2020 年真实选举结果(51.3\%)仅相差 0.2 个百分点,展现出卓越的模拟精度。相比之下,gpt-4o-mini 模型存在约 7\% 的 Biden 偏向,这可能与较小模型在角色扮演任务中的中立性不足有关。值得注意的是,在两个模型的模拟中,立场坚定的保守派群体(如虔诚基督徒、农村白人男性)均始终保持对 Trump 的支持,体现了 LLM 对不同政治画像的准确把握。

\subsubsection{投票演变过程}

图~\ref{fig:vote-evolution} 展示了 gpt-4o 模型在 6 轮辩论中的投票变化趋势。

\begin{figure}[H]
\centering
\includegraphics[width=0.9\textwidth]{figure/fig_vote_evolution.png}
\caption{gpt-4o 模型投票演变过程}
\label{fig:vote-evolution}
\end{figure}

\begin{table}[H]
\centering
\caption{各轮投票变化数据(gpt-4o 模型)}
\renewcommand{\arraystretch}{1.3}
\begin{tabular}{@{}clccc@{}}
\toprule
\textbf{轮次} & \textbf{议题} & \textbf{Biden} & \textbf{Trump} & \textbf{Undecided} \\
\midrule
初始 & --- & 35 & 33 & 33 \\
R1 & 经济政策 & 43 & 32 & 26 \\
R2 & 边境安全 & 46 & 41 & 14 \\
R3 & 疫情应对 & 49 & 44 & 8 \\
R4 & 法律秩序 & 52 & 46 & 3 \\
R5 & 宗教自由 & 53 & 46 & 2 \\
R6 & 中国贸易 & 52 & 46 & 3 \\
\bottomrule
\end{tabular}
\end{table}

%----------------------------------------
\subsection{实验二:CoT vs ToT 推理对比}
%----------------------------------------

\subsubsection{实验配置}

本实验旨在对比两种推理机制的决策质量差异。实验设置 30 位选民智能体,进行 3 轮辩论模拟。对比方法包括 Chain of Thought(CoT,线性推理)和 Tree of Thought(ToT,树状推理)两种策略。评估指标采用决策一致性,即相同政治倾向的选民群体在投票选择上的一致程度。

\subsubsection{实验结果}

\begin{table}[H]
\centering
\caption{CoT vs ToT 对比结果}
\renewcommand{\arraystretch}{1.3}
\begin{tabular}{@{}lcc@{}}
\toprule
\textbf{评估指标} & \textbf{CoT(线性推理)} & \textbf{ToT(树状推理)} \\
\midrule
Biden 最终票数 & 18 & 19 \\
Trump 最终票数 & 12 & 11 \\
Liberal 一致性 & 100\% & 100\% \\
Conservative 一致性 & 100\% & 100\% \\
Moderate 一致性 & 90\% & \textbf{100\%} \\
\bottomrule
\end{tabular}
\end{table}

实验结果表明,ToT 机制在中间派(Moderate)选民的决策一致性上实现了 10\% 的显著提升,从 CoT 的 90\% 提高至 100\%。这一改进源于 ToT 的多路径探索能力,使智能体能够更全面地考虑经济、社会、安全等不同维度的因素后做出决策。对于立场坚定的自由派(Liberal)和保守派(Conservative)选民,两种推理机制的表现相当,均达到 100\% 的一致性,表明这两类群体的决策受推理深度的影响较小。

%----------------------------------------
\subsection{实验三:资源分配协商}
%----------------------------------------

\subsubsection{实验配置}

本实验模拟了多主体资源分配协商场景。实验构建了包含 30 个智能体的系统,智能体角色涵盖医院、学校、工厂、办公楼和居民区等不同类型的资源需求方。系统中可分配的总资源量为 600 单位,最大协商轮数限制为 4 轮。实验对比了规则型协商策略与 LLM 驱动协商策略的分配效果差异。

\subsubsection{实验结果}

\begin{figure}[H]
\centering
\includegraphics[width=0.85\textwidth]{figure/fig_resource_comparison.png}
\caption{资源分配方案公平性对比}
\label{fig:resource-comparison}
\end{figure}

\begin{table}[H]
\centering
\caption{资源分配对比结果}
\renewcommand{\arraystretch}{1.3}
\begin{tabular}{@{}lcc@{}}
\toprule
\textbf{评估指标} & \textbf{规则型基线} & \textbf{LLM 驱动} \\
\midrule
协商轮数 & 2 & 4 \\
基尼系数 & 0.329 & \textbf{0.298} \\
平均满足率 & 84.9\% & \textbf{87.2\%} \\
最低满足率 & 43.8\% & \textbf{52.1\%} \\
资源利用率 & 98.2\% & 99.5\% \\
\bottomrule
\end{tabular}
\end{table}

实验数据显示,LLM 驱动的协商策略在公平性指标上显著优于规则型基线。基尼系数从 0.329 降至 0.298,降幅达 9.4\%,表明资源分配更加均衡。平均满足率和最低满足率分别提升至 87.2\% 和 52.1\%,后者相比基线提升了 19\%。这一结果表明 LLM 能够理解不同角色的紧迫性和优先级,在协商过程中为医院等关键机构分配更多资源,同时保障弱势群体的基本需求。

%----------------------------------------
\subsection{实验四:信息传播与判断}
%----------------------------------------

\subsubsection{实验配置}

本实验探究 LLM 驱动的智能体在社交网络信息传播场景中的判断能力。实验构建了包含 50 个智能体的社交网络,智能体类型包括普通用户、怀疑者、易信者和影响者四类。网络拓扑采用 Barabási-Albert 无标度网络模型,以模拟真实社交网络的度分布特征。实验向网络中注入 8 条信息,其中 5 条为真实信息,3 条为虚假信息,共进行 6 轮传播模拟。

\subsubsection{实验结果}

\begin{table}[H]
\centering
\caption{信息传播对比结果}
\renewcommand{\arraystretch}{1.3}
\begin{tabular}{@{}lcc@{}}
\toprule
\textbf{评估指标} & \textbf{规则型基线} & \textbf{LLM 驱动} \\
\midrule
整体判断准确率 & 80.0\% & 80.0\% \\
虚假信息识别率 & 0.0\% & 0.0\% \\
传播轮数 & 4 & 4 \\
\bottomrule
\end{tabular}
\end{table}

本实验中两种方法的表现相当,均未能有效识别虚假信息。分析表明,这一结果主要源于实验参数设置的局限性:初始传播者数量较少(仅 5 个)且虚假信息的可信度设置偏低,导致虚假信息在传播初期即被抑制,未能形成有效扩散。后续研究需调整初始传播者比例和信息可信度参数,以更充分地验证 LLM 在虚假信息识别方面的潜在优势。

%----------------------------------------
\subsection{案例分析:智能体推理过程}
%----------------------------------------

以下展示一位中间派选民(Moderate)在 CoT 推理下的决策过程:

\begin{tcolorbox}[
    colback=blue!5!white,
    colframe=blue!50!black,
    title={\textbf{Voter \#4} — 中间派选民(Swing Suburban)},
    fonttitle=\bfseries,
    boxrule=0.5pt,
    arc=2mm
]
\small
\textbf{背景}:郊区居民,关心医疗费用和就业问题\\[0.3em]
\textbf{初始立场}:Undecided

\tcblower
\textbf{【步骤1-分析辩论】}\\
这场辩论中,Biden 强调种族正义和尊重所有权利,而 Trump 则专注于法律与秩序和社区安全。

\textbf{【步骤2-评估候选人】}\\
Biden 的优点在于他对社会正义的关注,可能有助于实现更平等的医疗和就业机会;Trump 的优点在于他强调的法律与秩序可能会让某些社区感到更加安全。

\textbf{【步骤3-考虑个人议题】}\\
对于我关心的医疗费用和就业问题,Biden 似乎更有可能推动相关的改革和政策。

\textbf{【步骤4-最终决定】}\\
\textcolor{blue}{\textbf{Biden}}
\end{tcolorbox}

%========================================
\section{与 Proposal 对照分析}
%========================================

\subsection{完成度对照}

\begin{table}[H]
\centering
\caption{Proposal 计划与实际完成度对照}
\renewcommand{\arraystretch}{1.3}
\begin{tabularx}{\textwidth}{@{}>{\raggedright}Xcc>{\raggedright\arraybackslash}X@{}}
\toprule
\textbf{Proposal 内容} & \textbf{计划} & \textbf{实际} & \textbf{备注} \\
\midrule
多层次推理机制 (ToT) & $\checkmark$ & $\checkmark$ & 已实现并验证 \\
记忆检索优化 & $\checkmark$ & $\checkmark$ & 上下文感知、时间衰减 \\
反思机制优化 & $\checkmark$ & $\checkmark$ & 置信度触发 \\
协同决策机制 & $\checkmark$ & $\checkmark$ & 分布式+集中式 \\
选举实验 (101人, 6轮) & $\checkmark$ & $\checkmark$ & 偏差仅 0.2\% \\
资源分配实验 (50人) & $\checkmark$ & $\checkmark$ & 30人版本 \\
信息传播实验 (200节点) & $\checkmark$ & $\triangle$ & 50人版本 \\
CoT vs ToT 对比 & $\checkmark$ & $\checkmark$ & 新增实验 \\
\bottomrule
\end{tabularx}
\end{table}

\subsection{偏差与调整}

相较于 Proposal 的原定计划,实际执行过程中进行了若干调整。在实验规模方面,为有效控制 API 调用成本,资源分配实验的智能体数量从计划的 50 个缩减至 30 个,信息传播实验从 200 个节点缩减至 50 个。在实验内容方面,新增了 CoT 与 ToT 推理机制的直接对比实验,以更精确地验证 Proposal 中关于树状思维优越性的核心假设。此外,实验还引入了 gpt-4o 与 gpt-4o-mini 的多模型对比测试,发现模型能力的差异对模拟结果具有显著影响,这一发现为后续研究中的模型选择提供了重要参考。

%========================================
\section{主要发现与结论}
%========================================

\subsection{核心发现}

本研究获得了四项核心发现。首先,实验证明 LLM 能够有效模拟人类投票行为,gpt-4o 模型在选举模拟中实现了与真实选举仅 0.2\% 的偏差,充分验证了 LLM 驱动社会模拟的可行性和准确性。其次,Tree of Thought 推理机制显著提升了决策一致性,通过多路径探索使中间派选民的决策更加稳定,一致性从 CoT 的 90\% 提升至 100\%。第三,LLM 驱动的协商策略能够产生更公平的资源分配方案,这得益于 LLM 对不同角色紧迫性和优先级的深度理解,最终产生基尼系数更低、最低满足率更高的分配结果。最后,研究发现不同规模的 LLM 在角色扮演任务中表现差异显著,gpt-4o 相比 gpt-4o-mini 展现出更强的中立性和准确性。

\subsection{待解决问题}

当前研究仍存在若干待解决的问题。在实验设计层面,信息传播实验的参数配置需要进一步优化,以确保虚假信息具有足够的传播机会,从而更准确地评估 LLM 的判断能力。在系统可扩展性层面,大规模实验(超过 100 个智能体)的 API 调用成本控制仍是一个挑战,需要探索批量处理、缓存机制等优化策略。在框架集成层面,ToT 模块与现有 Casevo 框架的深度集成工作尚待完成,以实现更便捷的调用接口和更高效的运行性能。

%========================================
\section{下一步计划}
%========================================

后续研究将从四个方向推进。在实验优化方面,计划调整信息传播实验的初始传播者比例和信息可信度参数,以提升实验的区分度和说服力。在规模验证方面,将选举模拟场景的智能体数量扩展至 200 个以上,系统性验证框架的可扩展性和稳定性。在框架集成方面,将完成 ToT 模块与 Casevo Agent 基类的深度集成工作,提供统一的调用接口。在成果输出方面,将整理所有实验数据和分析结论,按照学术论文规范撰写最终研究报告。

\vspace{1cm}
\begin{center}
\rule{10cm}{0.5pt}\\[0.5cm]
\textit{报告完成于 \today}
\end{center}

\end{document}

