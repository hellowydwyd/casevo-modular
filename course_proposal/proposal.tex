\documentclass[12pt,a4paper]{ctexart}
\usepackage{geometry}
\usepackage{graphicx}
\usepackage{xcolor}
\usepackage{titlesec}
\usepackage{hyperref}
\usepackage{tikz}
\usepackage{pgfplots}
\usepackage{float}
\usepackage{caption}
\usepackage{subcaption}
\usepackage{amsmath}
\usepackage{amssymb}
\usepackage{fancyhdr}

\pgfplotsset{compat=1.18}
\usetikzlibrary{shapes,arrows,positioning,fit,backgrounds,calc,chains,mindmap,trees}

% 页面设置
\geometry{left=2.5cm,right=2.5cm,top=3cm,bottom=2.5cm,headheight=15pt}

% 页眉页脚设置
\pagestyle{fancy}
\fancyhf{}
\fancyhead[L]{人工智能与社会课程大作业}
\fancyhead[R]{选题意向书}
\fancyfoot[C]{\thepage}
\renewcommand{\headrulewidth}{0.4pt}
\renewcommand{\footrulewidth}{0pt}

% 超链接设置
\hypersetup{
    colorlinks=true,
    linkcolor=blue,
    filecolor=magenta,      
    urlcolor=cyan,
}

% 标题格式设置
\titleformat{\section}{\Large\bfseries\color{blue!70!black}}{\thesection}{1em}{}
\titleformat{\subsection}{\large\bfseries}{\thesubsection}{1em}{}

% 文档信息
\title{\textbf{\Huge 人工智能与社会课程\\[0.5cm]大作业选题意向书}}
\author{}
\date{}

\begin{document}

\maketitle
\thispagestyle{empty}

\vspace{0.5cm}

\begin{center}
\begin{tabular}{rl}
\textbf{题\quad 目:} & \large 基于 Casevo 框架的智能体决策能力优化研究 \\[0.4cm]
\textbf{小组成员:} & \large 王宇东(组长)、陈文远 \\[0.4cm]
\textbf{日\quad 期:} & \large \today \\
\end{tabular}
\end{center}

\vspace{1cm}

\begin{abstract}
\noindent
随着人工智能技术的快速发展,基于大语言模型的多智能体社会模拟系统成为研究社会演化、集体决策和复杂社会现象的重要工具。本研究以开源框架 Casevo 为基础,聚焦于智能体决策能力的优化问题。通过引入多层次推理机制、优化记忆检索策略、改进反思算法以及增强协同决策能力,我们期望显著提升智能体在复杂社会场景中的决策质量和适应性。本意向书详细阐述了研究背景、技术路线、实验设计和预期成果,为后续研究工作奠定基础。
\end{abstract}

\newpage
\setcounter{page}{1}

\section{拟定题目}

\textbf{基于 Casevo 框架的智能体决策能力优化研究}

\section{研究背景与意义}

\subsection{多智能体社会模拟的研究现状}

社会模拟作为计算社会科学的重要分支,长期以来致力于通过构建计算模型来理解和预测复杂的社会现象。传统的基于规则的智能体(Agent-based Model, ABM)虽然在特定场景下表现出色,但其行为逻辑高度依赖于预设规则,难以捕捉人类决策的复杂性和不确定性。近年来,大语言模型(Large Language Models, LLMs)的突破性进展为社会模拟带来了新的可能性。通过将 LLM 与多智能体系统相结合,研究者能够构建具有类人认知能力的虚拟智能体,这些智能体不仅能够理解自然语言,还能进行推理、记忆和反思,从而更真实地模拟人类在社会环境中的行为。

Casevo(Cognitive Agents and Social Evolution Simulator)正是在这一背景下应运而生的开源框架。该框架由 Jiang 等人于 2024 年提出,专门用于构建基于复杂网络的社会模拟多智能体实验。Casevo 在 Mesa 框架的基础上进行了扩展,集成了大语言模型接口、思维链(Chain of Thought)推理机制、检索增强生成(RAG)的记忆系统以及可定制的反思机制。这些特性使得 Casevo 在模拟选举投票、舆论演化、信息传播等社会现象方面展现出显著优势。

\subsection{当前面临的挑战}

尽管 Casevo 框架已经展现出强大的能力,但在应对更加复杂和动态的社会场景时,其智能体的决策机制仍存在一些不足之处。具体而言,现有的基于线性思维链的决策方式在处理多约束、多目标的复杂决策问题时,往往表现出较大的波动性,难以保证决策的稳定性和一致性。此外,当智能体需要处理大量历史信息时,现有的记忆检索机制在效率和准确性方面还有待提升。在多智能体协同场景中,智能体之间的信息交换和决策协商机制也相对简单,限制了群体智能的发挥。

\begin{figure}[H]
\centering
\includegraphics[width=0.85\textwidth]{images/Figure2Thearchitectureofthesystem.png}
\caption{Casevo 系统架构图(引自原论文 \cite{jiang2024casevo})。该架构包含四个核心模块:Model Module(模型模块)负责场景设置和事件调度,Agent Module(智能体模块)管理智能体行为和记忆,Parallel Optimization Module(并行优化模块)提高计算效率,Network Module(网络模块)构建和管理社会网络结构。}
\label{fig:casevo-arch}
\end{figure}

\begin{figure}[H]
\centering
\includegraphics[width=0.75\textwidth]{images/Figure1SchematicimplementationofCasevorounds.png}
\caption{Casevo 轮次执行机制示意图(引自原论文 \cite{jiang2024casevo})。系统采用基于轮次的离散事件模拟,每轮包含公共事件、智能体交互和状态更新等阶段,确保行为同步和事件有序调度。}
\label{fig:casevo-rounds}
\end{figure}

\subsection{研究意义}

本研究针对上述挑战,提出对 Casevo 框架中智能体决策能力的系统性优化方案。研究的理论意义在于探索大语言模型在社会模拟中的深层应用,推动认知智能体理论的发展;实践意义在于提供一套可落地的优化方法和工具,为社会科学研究者提供更强大的模拟平台,帮助他们更准确地预测和理解复杂社会现象。

\section{研究内容}

本研究将从理论研究、系统实现和实验验证三个层面展开,形成一个完整的研究闭环。

\subsection{多层次推理机制研究}

传统的思维链(Chain of Thought)方法采用线性的推理结构,智能体按照固定的顺序执行一系列推理步骤。这种方法在处理简单问题时效率较高,但面对复杂的决策场景时,线性结构的局限性就显现出来了。为了克服这一问题,我们计划引入树状思维(Tree of Thought, ToT)机制。与线性思维链不同,树状思维允许智能体在每个决策节点上探索多个可能的推理路径,通过对比不同路径的结果来选择最优方案。

\begin{figure}[H]
\centering
\begin{tikzpicture}[
    level 1/.style={sibling distance=4cm, level distance=2cm},
    level 2/.style={sibling distance=2cm, level distance=2cm},
    node/.style={circle, draw, minimum size=0.8cm}
]
    % CoT部分
    \begin{scope}[xshift=-4cm]
        \node[node, fill=blue!20] (cot1) {1}
            child {node[node, fill=blue!20] (cot2) {2}
                child {node[node, fill=blue!20] (cot3) {3}
                    child {node[node, fill=green!30] (cot4) {结果}}
                }
            };
        \node[above of=cot1, yshift=0.3cm] {\textbf{Chain of Thought}};
        \node[below of=cot4, yshift=-0.5cm] {线性推理};
    \end{scope}
    
    % ToT部分
    \begin{scope}[xshift=4cm]
        \node[node, fill=orange!20] (tot1) {1}
            child {node[node, fill=orange!20] (tot2a) {2a}
                child {node[node, fill=green!30] (tot3a) {3a}}
                child {node[node] (tot3b) {3b}}
            }
            child {node[node, fill=orange!20] (tot2b) {2b}
                child {node[node] (tot3c) {3c}}
                child {node[node, fill=green!30] (tot3d) {3d}}
            };
        \node[above of=tot1, yshift=0.3cm] {\textbf{Tree of Thought}};
        \node[below of=tot3b, yshift=-0.5cm] {多路径探索};
    \end{scope}
\end{tikzpicture}
\caption{思维链与树状思维的对比}
\label{fig:cot-vs-tot}
\end{figure}

具体而言,我们将设计一个分层决策架构,在该架构中,智能体首先对问题进行初步分析,识别出关键的决策维度,然后针对每个维度生成多个候选方案。通过对这些候选方案进行评估和筛选,智能体最终选择综合评分最高的方案作为决策结果。这种方法不仅能够提高决策的质量,还能增强决策的鲁棒性,因为即使某一路径上出现错误,其他路径仍可能导向正确的结果。

此外,我们还将研究推理深度对决策质量的影响。过浅的推理可能导致决策考虑不周,而过深的推理则会增加计算成本。因此,我们需要找到一个平衡点,使得智能体能够在有限的计算资源下做出高质量的决策。

\subsection{记忆检索与利用策略优化}

记忆系统是认知智能体的核心组件之一。Casevo 当前使用基于 ChromaDB 的向量数据库来存储和检索智能体的记忆。虽然向量检索在相似度匹配方面表现出色,但在处理大规模记忆数据时,检索效率和准确性仍有提升空间。我们计划从以下几个方面对记忆系统进行优化。

首先,我们将引入上下文感知的记忆筛选机制。传统的向量检索主要基于语义相似度,但在实际应用中,记忆的相关性不仅取决于内容的相似性,还取决于时间、情境等因素。例如,在选举投票场景中,最近发生的辩论事件通常比几周前的事件更具参考价值。因此,我们将在相似度计算中引入时间衰减因子和情境匹配度,使得检索结果更加符合当前决策的需要。

其次,我们将研究短期记忆与长期记忆的协同利用策略。短期记忆存储最近的交互信息,具有较高的时效性;长期记忆则存储经过反思和总结的稳定观点,具有较强的一致性。如何在决策过程中合理平衡两者的权重,是一个值得深入探讨的问题。我们计划设计一个动态权重分配机制,根据决策任务的性质自动调整短期记忆和长期记忆的贡献比例。

最后,为了应对大规模记忆数据带来的存储压力,我们将实现智能记忆压缩与遗忘机制。该机制能够识别出低重要性的记忆,并将其从数据库中删除或归档,从而保持记忆系统的精简和高效。

\begin{figure}[H]
\centering
\includegraphics[width=0.6\textwidth]{images/Figure3Thestructureofthenetworkbetweenvoters.png}
\caption{选民社会网络结构示意图(引自原论文 \cite{jiang2024casevo})。该网络采用小世界网络拓扑,包含 101 个节点,每个节点代表一个选民智能体,边表示选民之间的社交关系。这种网络结构能够真实模拟现实社会中的信息传播和意见交流模式。}
\label{fig:voter-network}
\end{figure}

\subsection{反思机制优化}

反思(Reflection)是智能体自我改进的重要手段。通过定期回顾历史经验,智能体能够提取出一般性的规律和见解,从而在未来的决策中做出更明智的选择。Casevo 已经集成了基于 Chain of Thought 的反思机制,但该机制的触发条件和反思深度都是固定的,缺乏灵活性。

我们计划引入元认知机制,使智能体能够评估自身决策的可信度。具体而言,智能体在做出决策后,不仅输出决策结果,还输出一个置信度分数。当置信度较低时,智能体会自动触发反思过程,重新审视自己的推理逻辑和所依据的记忆信息,并尝试修正可能的错误。这种动态反思策略能够使智能体在保证决策效率的同时,避免因信息不足或推理错误导致的严重失误。

此外,我们还将设计多层次的反思机制。浅层反思关注具体决策的对错,而深层反思则关注决策背后的价值观和信念。通过结合这两个层次的反思,智能体能够形成更加稳定和一致的长期观点。

\subsection{多智能体协同决策机制}

在社会模拟中,智能体之间的交互和协作是不可或缺的环节。当前 Casevo 支持智能体之间的对话和信息交换,但缺乏系统化的协同决策机制。我们计划设计一套完整的协同决策框架,包括信息交换协议、决策协商机制和结果聚合算法。

在信息交换方面,我们将定义标准化的消息格式,使得智能体能够清晰地表达自己的观点、依据和置信度。在决策协商方面,我们将借鉴共识算法的思想,设计一个迭代式的协商过程。在每一轮协商中,智能体交换彼此的观点,并根据其他智能体的意见调整自己的立场。经过多轮协商后,群体逐渐形成共识,或者识别出无法调和的分歧。

\begin{figure}[H]
\centering
\begin{tikzpicture}[
    agent/.style={circle, draw, fill=blue!30, minimum size=1.2cm, font=\small},
    central/.style={circle, draw, fill=red!30, minimum size=1cm, font=\small}
]
    % 分布式决策
    \begin{scope}[xshift=-4cm]
        \node[agent] (a1) at (0,2) {A1};
        \node[agent] (a2) at (2,1) {A2};
        \node[agent] (a3) at (2,-1) {A3};
        \node[agent] (a4) at (0,-2) {A4};
        \node[agent] (a5) at (-2,-1) {A5};
        \node[agent] (a6) at (-2,1) {A6};
        
        \draw[<->, thick] (a1) -- (a2);
        \draw[<->, thick] (a2) -- (a3);
        \draw[<->, thick] (a3) -- (a4);
        \draw[<->, thick] (a4) -- (a5);
        \draw[<->, thick] (a5) -- (a6);
        \draw[<->, thick] (a6) -- (a1);
        
        \node[below of=a4, yshift=-0.5cm] {\textbf{分布式协商}};
    \end{scope}
    
    % 集中式决策
    \begin{scope}[xshift=4cm]
        \node[central] (c) at (0,0) {C};
        \node[agent] (b1) at (0,2) {A1};
        \node[agent] (b2) at (1.7,1) {A2};
        \node[agent] (b3) at (1.7,-1) {A3};
        \node[agent] (b4) at (0,-2) {A4};
        \node[agent] (b5) at (-1.7,-1) {A5};
        \node[agent] (b6) at (-1.7,1) {A6};
        
        \draw[->, thick] (b1) -- (c);
        \draw[->, thick] (b2) -- (c);
        \draw[->, thick] (b3) -- (c);
        \draw[->, thick] (b4) -- (c);
        \draw[->, thick] (b5) -- (c);
        \draw[->, thick] (b6) -- (c);
        
        \node[below of=c, yshift=-2.5cm] {\textbf{集中式聚合}};
    \end{scope}
\end{tikzpicture}
\caption{多智能体协同决策模式对比}
\label{fig:collaborative-decision}
\end{figure}

在结果聚合方面,我们将探索分布式决策与集中式决策的权衡。分布式决策允许每个智能体保持独立性,适合处理观点多元的场景;集中式决策则通过一个中心节点来整合所有智能体的意见,适合需要快速达成一致的场景。我们将根据不同任务的特点,设计自适应的决策模式选择策略。

\subsection{系统实现}

理论研究需要通过系统实现来验证其可行性和有效性。我们将基于 Casevo 框架进行扩展开发,实现上述各项优化方案。具体而言,我们将开发以下几个核心模块:增强型思维链模块(\texttt{enhanced\_chain.py}),支持树状推理和多路径探索;优化的记忆管理模块(\texttt{advanced\_memory.py}),集成上下文感知检索和智能遗忘机制;协同决策模块(\texttt{collaborative\_decision.py}),提供信息交换和协商框架;决策质量评估模块(\texttt{decision\_evaluator.py}),用于量化评估决策的质量和可信度。

所有代码将遵循 Casevo 项目的编码规范,确保与现有框架(v0.3.19)完全兼容。同时,我们将提供详细的 API 文档和使用示例,方便其他研究者使用和扩展我们的工作。

\subsection{实验验证}

为了全面评估优化方案的效果,我们将设计一系列严谨的对比实验。实验方案遵循科学实验的基本原则,采用控制变量法,确保结果的可靠性和可重复性。

\textbf{实验场景设计。}我们将构建三个具有代表性的社会模拟场景,每个场景针对不同的决策挑战。第一个是选举投票场景,该场景基于 2020 年美国总统大选的辩论内容,包含六轮辩论的完整文本资料。我们将配置 101 个具有不同政治倾向的选民智能体,按照 Pew Research Center 的政治类型学分为九个类别。智能体在小世界网络中相互连接,网络平均度数为 6,聚类系数约为 0.3,模拟真实的社交关系。每轮辩论后,智能体将观看辩论内容、与邻居讨论、进行反思,最终投票。我们将记录每轮的投票变化、意见演化轨迹以及网络中的信息传播路径。

第二个是资源分配场景,模拟 50 个智能体在资源受限情况下的协商过程。总资源量固定为 1000 单位,每个智能体有不同的资源需求(范围 15-30 单位)和优先级权重。智能体需要通过多轮协商来达成分配方案,既要满足自身需求,也要考虑整体公平性。我们将评估协商轮次、达成共识的速度以及最终分配方案的公平性指标(基尼系数、方差等)。

第三个是信息传播场景,研究虚假信息在社交网络中的扩散动力学。我们构建一个包含 200 个节点的无标度网络,少数节点(10\%)初始接收到虚假信息。智能体需要根据信息来源的可信度、内容的逻辑一致性以及与自身认知的匹配度来判断信息真伪。我们将追踪虚假信息的传播范围、传播速度以及不同智能体决策机制对抑制虚假信息的有效性。

\textbf{实验参数配置。}对于每个场景,我们将设置详细的实验参数。在选举场景中,LLM 温度参数设为 0.7(平衡创造性和稳定性),记忆检索返回前 5 条最相关记忆,反思触发阈值为置信度低于 0.6。在资源分配场景中,协商最大轮次设为 10 轮,每轮允许智能体修改自己的需求提议,收敛判定标准为连续两轮所有智能体的提议变化小于 5\%。在信息传播场景中,信息传播概率与边权重成正比(范围 0.3-0.7),智能体判断阈值根据个体特征有所差异。

\textbf{基线方法与对比实验。}我们设置三组对照实验:基线组使用原始 Casevo 框架的 Chain of Thought 决策机制;优化组 A 仅应用 Tree of Thought 多层次推理;优化组 B 综合应用所有优化方案(ToT + 增强记忆 + 动态反思 + 协同决策)。每组实验独立运行 10 次,使用不同的随机种子,以排除随机因素的影响。我们将采集每次运行的完整日志,包括决策轨迹、记忆检索记录、推理过程以及最终结果。

\begin{figure}[H]
\centering
\begin{subfigure}[b]{0.48\textwidth}
    \centering
    \includegraphics[width=\textwidth]{images/(a)VotingResult.png}
    \caption{选举投票结果变化趋势}
    \label{fig:voting-result}
\end{subfigure}
\hfill
\begin{subfigure}[b]{0.48\textwidth}
    \centering
    \includegraphics[width=\textwidth]{images/(b)Wordcloudofvoteropinions.png}
    \caption{选民意见词云分析}
    \label{fig:wordcloud}
\end{subfigure}
\caption{Casevo 框架选举模拟实验结果(引自原论文 \cite{jiang2024casevo})。左图展示了六轮辩论中选民对两位候选人支持度的动态变化,可以观察到 Biden 的支持率稳步上升,Trump 的支持率有所波动,中立选民比例逐渐减少。右图通过词云展示选民对候选人的意见分布,Biden 相关的词汇以"支持"、"赞同"等正面词汇为主,而 Trump 相关的词汇则呈现正负两极分化的特点。}
\label{fig:experiment-results}
\end{figure}

对比实验将采用控制变量法,基线方法使用原始 Casevo 框架的决策机制,优化方法则集成我们提出的各项改进。评估指标涵盖四个维度:决策质量(准确率、一致性、合理性)、推理能力(深度、多样性、连贯性)、计算效率(响应时间、内存占用、可扩展性)以及社会效应(群体共识度、意见分化程度、社会稳定性)。通过多维度的量化评估,我们能够全面了解优化方案的优势和局限,为后续改进提供依据。

\section{预期成果}

本研究预期在以下三个方面产出高质量的成果。

\textbf{优化后的 Casevo 框架代码。}我们将开发一套完整的决策能力优化模块,包括增强型思维链、优化的记忆管理、协同决策和决策质量评估等核心组件。所有代码将开源发布,并确保与现有 Casevo 框架完全兼容。我们还将提供详细的 API 文档和使用教程,帮助研究者快速上手。代码的开发将严格遵循软件工程规范,包括单元测试、代码审查和版本控制,以保证代码的质量和可维护性。

\textbf{详细实验报告。}实验报告将包含完整的实验设计说明、数据采集过程、对比实验结果以及深入的案例分析。我们将使用丰富的图表来可视化实验数据,使得结果更加直观易懂。报告还将讨论实验过程中遇到的问题和挑战,以及我们是如何解决这些问题的。这些经验和教训对于后续研究具有重要的参考价值。

\textbf{性能对比分析。}我们将提供全面的性能评估报告,从决策质量、推理能力、计算效率和社会效应四个维度对优化方案进行量化分析。通过与基线方法的详细对比,我们将明确指出各项优化的具体贡献度,以及在不同场景下的适用性。此外,我们还将进行敏感性分析,探讨关键参数(如推理深度、记忆容量、协商轮次等)对系统性能的影响,为参数调优提供指导。

\section{小组成员及分工}

本研究由两名成员组成,分工明确、协作紧密。

\begin{center}
\begin{tabular}{|c|c|p{9cm}|}
\hline
\textbf{姓名} & \textbf{角色} & \textbf{主要职责} \\
\hline
王宇东 & 组长 & 负责项目整体规划与协调,深入开展理论研究,设计优化算法,撰写研究报告和学术论文,确保研究的学术质量和创新性 \\
\hline
陈文远 & 组员 & 负责代码实现与系统集成,设计和执行实验,进行数据分析与可视化,协助报告撰写,保证研究成果的技术质量和可复现性 \\
\hline
\end{tabular}
\end{center}

\vspace{0.5cm}

两位成员将采用敏捷开发的协作模式,定期召开小组会议讨论研究进展和遇到的技术难题。所有代码通过 Git 进行版本管理,确保开发过程的可追溯性。在研究过程中,双方将相互审阅对方的工作成果,通过同行评审机制提高研究质量。对于重要的技术决策和理论创新,两位成员将共同讨论并达成一致意见后再推进实施。

\section{时间规划}

本研究计划在 10 周内完成,具体安排如下。

\begin{figure}[H]
\centering
\begin{tikzpicture}
    \begin{axis}[
        xbar,
        width=14cm,
        height=8cm,
        enlargelimits=0.15,
        xlabel={周次},
        symbolic y coords={第五阶段,第四阶段,第三阶段,第二阶段,第一阶段},
        ytick=data,
        nodes near coords,
        nodes near coords align={horizontal},
        bar width=0.6cm,
    ]
    \addplot coordinates {(2,第一阶段) (4,第二阶段) (3,第三阶段) (2,第四阶段) (1,第五阶段)};
    \end{axis}
\end{tikzpicture}
\caption{研究时间安排甘特图}
\label{fig:gantt}
\end{figure}

\begin{center}
\begin{tabular}{|c|c|p{8cm}|}
\hline
\textbf{阶段} & \textbf{周期} & \textbf{主要任务} \\
\hline
第一阶段 & 第6-7周 & 文献调研与理论研究,深入阅读相关论文,分析现有方法的优缺点;完成技术方案设计,确定具体的优化策略和实现路径 \\
\hline
第二阶段 & 第8-11周 & 代码实现与模块开发,完成增强型思维链、优化记忆管理等核心模块的编码工作;进行单元测试,确保各模块功能正确 \\
\hline
第三阶段 & 第12-14周 & 实验设计与数据采集,构建实验场景,配置参数,运行基线实验和优化实验;进行初步数据分析,验证方案可行性 \\
\hline
第四阶段 & 第15-16周 & 对比实验与性能评估,完成所有计划实验,收集完整数据;进行深入的结果分析,总结优化方案的优势和不足 \\
\hline
第五阶段 & 第17周 & 报告撰写与成果整理,完成实验报告和代码文档的编写;准备成果展示材料,包括幻灯片和演示视频 \\
\hline
\end{tabular}
\end{center}

整个研究过程将采用迭代式推进的方式,每个阶段结束时进行阶段性总结和评审。如果某一阶段的工作未能达到预期目标,我们将及时调整计划,确保最终成果的质量。

\section{参考文献}

\begin{thebibliography}{9}
    \bibitem{jiang2024casevo} Jiang Z, Shi Y, Li M, et al. Casevo: A Cognitive Agents and Social Evolution Simulator[J]. arXiv preprint arXiv:2412.19498, 2024.
    
    \bibitem{wei2022cot} Wei J, Wang X, Schuurmans D, et al. Chain-of-thought prompting elicits reasoning in large language models[J]. Advances in Neural Information Processing Systems, 2022, 35: 24824-24837.
    
    \bibitem{yao2024tot} Yao S, Yu D, Zhao J, et al. Tree of thoughts: Deliberate problem solving with large language models[J]. Advances in Neural Information Processing Systems, 2024, 36.
    
    \bibitem{park2023generative} Park J S, O'Brien J, Cai C J, et al. Generative agents: Interactive simulacra of human behavior[C]//Proceedings of the 36th Annual ACM Symposium on User Interface Software and Technology. 2023: 1-22.
    
    \bibitem{wilensky2015abm} Wilensky U, Rand W. An introduction to agent-based modeling: modeling natural, social, and engineered complex systems with NetLogo[M]. MIT Press, 2015.
    
    \bibitem{axelrod1997cooperation} Axelrod R. The complexity of cooperation: Agent-based models of competition and collaboration[M]. Princeton University Press, 1997.
    
    \bibitem{bonabeau2002abm} Bonabeau E. Agent-based modeling: Methods and techniques for simulating human systems[J]. Proceedings of the National Academy of Sciences, 2002, 99(suppl\_3): 7280-7287.
    
    \bibitem{mesa} Mesa: Agent-based modeling in Python 3+. \url{https://github.com/projectmesa/mesa}
    
    \bibitem{chromadb} ChromaDB: The AI-native open-source embedding database. \url{https://www.trychroma.com/}
\end{thebibliography}

\end{document}
